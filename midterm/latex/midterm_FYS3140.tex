\documentclass[12pt,twoside]{article}
\usepackage{cancel}
\usepackage{physics}
\usepackage{bm}
\usepackage{varioref}
\usepackage{booktabs}
\usepackage{commath}
\usepackage{amsmath}
\usepackage{tikz}
%\usepackage{cite}
\usetikzlibrary{shapes.misc}

\tikzset{cross/.style={cross out, draw=black, minimum size=2*(#1-\pgflinewidth), inner sep=0pt, outer sep=0pt},
%default radius will be 1pt.
cross/.default={1pt}}

\newcommand{\reporttitle}{Midterm exam}
\newcommand{\reportauthor}{}
\newcommand{\reporttype}{}
%\newcommand{\cid}{}

% include files that load packages and define macros
%%%%%%%%%%%%%%%%%%%%%%%%%%%%%%%%%%%%%%%%%
% University Assignment Title Page 
% LaTeX Template
% Version 1.0 (27/12/12)
%
% This template has been downloaded from:
% http://www.LaTeXTemplates.com
%
% Original author:
% WikiBooks (http://en.wikibooks.org/wiki/LaTeX/Title_Creation)
%
% License:
% CC BY-NC-SA 3.0 (http://creativecommons.org/licenses/by-nc-sa/3.0/)
% 
% Instructions for using this template:
% This title page is capable of being compiled as is. This is not useful for 
% including it in another document. To do this, you have two options: 
%
% 1) Copy/paste everything between \begin{document} and \end{document} 
% starting at \begin{titlepage} and paste this into another LaTeX file where you 
% want your title page.
% OR
% 2) Remove everything outside the \begin{titlepage} and \end{titlepage} and 
% move this file to the same directory as the LaTeX file you wish to add it to. 
% Then add \input{./title_page_1.tex} to your LaTeX file where you want your
% title page.
%
%----------------------------------------------------------------------------------------
%	PACKAGES AND OTHER DOCUMENT CONFIGURATIONS
%----------------------------------------------------------------------------------------
\usepackage{ifxetex}
\usepackage{textpos}
\usepackage{natbib}
\usepackage{kpfonts}
\usepackage[a4paper,hmargin=2.8cm,vmargin=2.0cm,includeheadfoot]{geometry}
\usepackage{ifxetex}
\usepackage{stackengine}
\usepackage{tabularx,longtable,multirow,subfigure,caption}%hangcaption
\usepackage{fncylab} %formatting of labels
\usepackage{fancyhdr}
\usepackage{color}
\usepackage[tight,ugly]{units}
\usepackage{url}
\usepackage{float}
\usepackage[english]{babel}
\usepackage{amsmath}
\usepackage{graphicx}
\usepackage[colorinlistoftodos]{todonotes}
\usepackage{dsfont}
\usepackage{epstopdf} % automatically replace .eps with .pdf in graphics
\usepackage{natbib}
\usepackage{backref}
\usepackage{array}
\usepackage{latexsym}
\usepackage{etoolbox}

\usepackage{enumerate} % for numbering with [a)] format 



\ifxetex
\usepackage{fontspec}
\setmainfont[Scale=.8]{OpenDyslexic-Regular}
\else
\usepackage[pdftex,pagebackref,hypertexnames=false,colorlinks]{hyperref} % provide links in pdf
\hypersetup{pdftitle={},
  pdfsubject={}, 
  pdfauthor={\reportauthor},
  pdfkeywords={}, 
  pdfstartview=FitH,
  pdfpagemode={UseOutlines},% None, FullScreen, UseOutlines
  bookmarksnumbered=true, bookmarksopen=true, colorlinks,
    citecolor=black,%
    filecolor=black,%
    linkcolor=black,%
    urlcolor=black}
\usepackage[all]{hypcap}
\fi

\usepackage{tcolorbox}

% various theorems
\usepackage{ntheorem}
\theoremstyle{break}
\newtheorem{lemma}{Lemma}
\newtheorem{theorem}{Theorem}
\newtheorem{remark}{Remark}
\newtheorem{definition}{Definition}
\newtheorem{proof}{Proof}

% example-environment
\newenvironment{example}[1][]
{ 
\vspace{4mm}
\noindent\makebox[\linewidth]{\rule{\hsize}{1.5pt}}
\textbf{Example #1}\\
}
{ 
\noindent\newline\makebox[\linewidth]{\rule{\hsize}{1.0pt}}
}



%\renewcommand{\rmdefault}{pplx} % Palatino
% \renewcommand{\rmdefault}{put} % Utopia

\ifxetex
\else
\renewcommand*{\rmdefault}{bch} % Charter
\renewcommand*{\ttdefault}{cmtt} % Computer Modern Typewriter
%\renewcommand*{\rmdefault}{phv} % Helvetica
%\renewcommand*{\rmdefault}{iwona} % Avant Garde
\fi

\setlength{\parindent}{0em}  % indentation of paragraph

\setlength{\headheight}{14.5pt}
\pagestyle{fancy}
\fancyfoot[ER,OL]{\thepage}%Page no. in the left on
                                %odd pages and on right on even pages
\fancyfoot[OC,EC]{\sffamily }
\renewcommand{\headrulewidth}{0.1pt}
\renewcommand{\footrulewidth}{0.1pt}
\captionsetup{margin=10pt,font=small,labelfont=bf}


%--- chapter heading

\def\@makechapterhead#1{%
  \vspace*{10\p@}%
  {\parindent \z@ \raggedright %\sffamily
        %{\Large \MakeUppercase{\@chapapp} \space \thechapter}
        %\\
        %\hrulefill
        %\par\nobreak
        %\vskip 10\p@
    \interlinepenalty\@M
    \Huge \bfseries 
    \thechapter \space\space #1\par\nobreak
    \vskip 30\p@
  }}

%---chapter heading for \chapter*  
\def\@makeschapterhead#1{%
  \vspace*{10\p@}%
  {\parindent \z@ \raggedright
    \sffamily
    \interlinepenalty\@M
    \Huge \bfseries  
    #1\par\nobreak
    \vskip 30\p@
  }}
  



% %%%%%%%%%%%%% boxit
\def\Beginboxit
   {\par
    \vbox\bgroup
	   \hrule
	   \hbox\bgroup
		  \vrule \kern1.2pt %
		  \vbox\bgroup\kern1.2pt
   }

\def\Endboxit{%
			      \kern1.2pt
		       \egroup
		  \kern1.2pt\vrule
		\egroup
	   \hrule
	 \egroup
   }	

\newenvironment{boxit}{\Beginboxit}{\Endboxit}
\newenvironment{boxit*}{\Beginboxit\hbox to\hsize{}}{\Endboxit}



\allowdisplaybreaks

\makeatletter
\newcounter{elimination@steps}
\newcolumntype{R}[1]{>{\raggedleft\arraybackslash$}p{#1}<{$}}
\def\elimination@num@rights{}
\def\elimination@num@variables{}
\def\elimination@col@width{}
\newenvironment{elimination}[4][0]
{
    \setcounter{elimination@steps}{0}
    \def\elimination@num@rights{#1}
    \def\elimination@num@variables{#2}
    \def\elimination@col@width{#3}
    \renewcommand{\arraystretch}{#4}
    \start@align\@ne\st@rredtrue\m@ne
}
{
    \endalign
    \ignorespacesafterend
}
\newcommand{\eliminationstep}[2]
{
    \ifnum\value{elimination@steps}>0\leadsto\quad\fi
    \left[
        \ifnum\elimination@num@rights>0
            \begin{array}
            {@{}*{\elimination@num@variables}{R{\elimination@col@width}}
            |@{}*{\elimination@num@rights}{R{\elimination@col@width}}}
        \else
            \begin{array}
            {@{}*{\elimination@num@variables}{R{\elimination@col@width}}}
        \fi
            #1
        \end{array}
    \right]
    & 
    \begin{array}{l}
        #2
    \end{array}
    &%                                    moved second & here
    \addtocounter{elimination@steps}{1}
}
\makeatother

%% Fast macro for column vectors
\makeatletter  
\def\colvec#1{\expandafter\colvec@i#1,,,,,,,,,\@nil}
\def\colvec@i#1,#2,#3,#4,#5,#6,#7,#8,#9\@nil{% 
  \ifx$#2$ \begin{bmatrix}#1\end{bmatrix} \else
    \ifx$#3$ \begin{bmatrix}#1\\#2\end{bmatrix} \else
      \ifx$#4$ \begin{bmatrix}#1\\#2\\#3\end{bmatrix}\else
        \ifx$#5$ \begin{bmatrix}#1\\#2\\#3\\#4\end{bmatrix}\else
          \ifx$#6$ \begin{bmatrix}#1\\#2\\#3\\#4\\#5\end{bmatrix}\else
            \ifx$#7$ \begin{bmatrix}#1\\#2\\#3\\#4\\#5\\#6\end{bmatrix}\else
              \ifx$#8$ \begin{bmatrix}#1\\#2\\#3\\#4\\#5\\#6\\#7\end{bmatrix}\else
                 \PackageError{Column Vector}{The vector you tried to write is too big, use bmatrix instead}{Try using the bmatrix environment}
              \fi
            \fi
          \fi
        \fi
      \fi
    \fi
  \fi 
}  
\makeatother

\robustify{\colvec}

%%% Local Variables: 
%%% mode: latex
%%% TeX-master: "notes"
%%% End: 
 % various packages needed for maths etc.
% quick way of adding a figure
\newcommand{\fig}[3]{
 \begin{center}
 \scalebox{#3}{\includegraphics[#2]{#1}}
 \end{center}
}

%\newcommand*{\point}[1]{\vec{\mkern0mu#1}}
\newcommand{\ci}[0]{\perp\!\!\!\!\!\perp} % conditional independence
\newcommand{\point}[1]{{#1}} % points 
\renewcommand{\vec}[1]{{\boldsymbol{{#1}}}} % vector
\newcommand{\mat}[1]{{\boldsymbol{{#1}}}} % matrix
\newcommand{\R}[0]{\mathds{R}} % real numbers
\newcommand{\Z}[0]{\mathds{Z}} % integers
\newcommand{\N}[0]{\mathds{N}} % natural numbers
\newcommand{\nat}[0]{\mathds{N}} % natural numbers
\newcommand{\Q}[0]{\mathds{Q}} % rational numbers
\ifxetex
\newcommand{\C}[0]{\mathds{C}} % complex numbers
\else
\newcommand{\C}[0]{\mathds{C}} % complex numbers
\fi
\newcommand{\tr}[0]{\text{tr}} % trace
\renewcommand{\d}[0]{\mathrm{d}} % total derivative
\newcommand{\inv}{^{-1}} % inverse
\newcommand{\id}{\mathrm{id}} % identity mapping
\renewcommand{\dim}{\mathrm{dim}} % dimension
\newcommand{\rank}[0]{\mathrm{rk}} % rank
\newcommand{\determ}[1]{\mathrm{det}(#1)} % determinant
\newcommand{\scp}[2]{\langle #1 , #2 \rangle}
\newcommand{\kernel}[0]{\mathrm{ker}} % kernel/nullspace
\newcommand{\img}[0]{\mathrm{Im}} % image
\newcommand{\idx}[1]{{(#1)}}
\DeclareMathOperator*{\diag}{diag}
\newcommand{\E}{\mathds{E}} % expectation
\newcommand{\var}{\mathds{V}} % variance
\newcommand{\gauss}[2]{\mathcal{N}\big(#1,\,#2\big)} % gaussian distribution N(.,.)
\newcommand{\gaussx}[3]{\mathcal{N}\big(#1\,|\,#2,\,#3\big)} % gaussian distribution N(.|.,.)
\newcommand{\gaussBig}[2]{\mathcal{N}\left(#1,\,#2\right)} % see above, but with brackets that adjust to the height of the arguments
\newcommand{\gaussxBig}[3]{\mathcal{N}\left(#1\,|\,#2,\,#3\right)} % see above, but with brackets that adjust to the height of the arguments
\DeclareMathOperator{\cov}{Cov} % covariance (matrix) 
\ifxetex
\renewcommand{\T}[0]{^\top} % transpose
\else
\newcommand{\T}[0]{^\top}
\fi
% matrix determinant
\newcommand{\matdet}[1]{
\left|
\begin{matrix}
#1
\end{matrix}
\right|
}



%%% various color definitions
\definecolor{darkgreen}{rgb}{0,0.6,0}

\newcommand{\blue}[1]{{\color{blue}#1}}
\newcommand{\red}[1]{{\color{red}#1}}
\newcommand{\green}[1]{{\color{darkgreen}#1}}
\newcommand{\orange}[1]{{\color{orange}#1}}
\newcommand{\magenta}[1]{{\color{magenta}#1}}
\newcommand{\cyan}[1]{{\color{cyan}#1}}


% redefine emph
\renewcommand{\emph}[1]{\blue{\bf{#1}}}

% place a colored box around a character
\gdef\colchar#1#2{%
  \tikz[baseline]{%
  \node[anchor=base,inner sep=2pt,outer sep=0pt,fill = #2!20] {#1};
    }%
}%
 % short-hand notation and macros


%%%%%%%%%%%%%%%%%%%%%%%%%%%%

\begin{document}
% front page
% Last modification: 2016-09-29 (Marc Deisenroth)
\begin{titlepage}

\newcommand{\HRule}{\rule{\linewidth}{0.5mm}} % Defines a new command for the horizontal lines, change thickness here


%----------------------------------------------------------------------------------------
%	LOGO SECTION
%----------------------------------------------------------------------------------------

%\includegraphics[width = 2.5cm]{./figures/UiO}\\[0.5cm]

\begin{center} % Center remainder of the page

%----------------------------------------------------------------------------------------
%	HEADING SECTIONS
%----------------------------------------------------------------------------------------
\textsc{\LARGE \reporttype}\\[1.5cm]
\textsc{\Large University of Oslo}\\[0.5cm]
\textsc{\large FYS3140 - Mathematical methods in physics}\\[0.5cm]
%----------------------------------------------------------------------------------------
%	TITLE SECTION
%----------------------------------------------------------------------------------------

\HRule \\[0.4cm]
{ \huge \bfseries \reporttitle}\\ % Title of your document
\HRule \\[1.5cm]
\end{center}
%----------------------------------------------------------------------------------------
%	AUTHOR SECTION
%----------------------------------------------------------------------------------------

%\begin{minipage}{0.4\hsize}
\begin{flushleft} \large
\textit{Candidate number: ---}\\
%\reportauthor~(CID: \cid) % Your name
\end{flushleft}
\vspace{2cm}
\makeatletter
Date: \@date

\vfill % Fill the rest of the page with whitespace



\makeatother


\end{titlepage}

\section*{Problem 1: Complex analysis}
\subsection*{Part A: Cauchy integral formula and harmonic functions}
We will begin by studying Cauchy's integral formula for a function $f(z)$ which is analytical inside and on the closed curve $C_R$ defined by
\begin{equation}
  C_R: \abs{z-z_0}=R \label{contour}
\end{equation}
Cauchy's integral formula takes the form
\begin{equation}
  f(z_0) = \frac{1}{2\pi i} \oint_{C_R} \frac{f(z)}{z-z_0}\dif z \label{CIF}
\end{equation}
\subsubsection*{a)}
We let $M$ denote the maximum absolute value of $f(z)$ on $C_R$ such that
\begin{equation}
  \abs{f(z)} \leq M, \label{M}
\end{equation}
for all $z$ on the contour. We will use this to find an upper bound on the absolute value of $f(z_0)$
\begin{equation}
  \abs{f(z_0)} = \abs{\frac{1}{2\pi i} \oint_{C_R} \frac{f(z)}{z-z_0}\dif z} = \frac{1}{2\pi}\,\abs{\oint_{C_R} \frac{f(z)}{z-z_0}\dif z}. \label{begining}
\end{equation}
We begin by writing the absolute value of the integral in terms of Riemann sums
\begin{equation}
  \abs{\oint_{C_R} \frac{f(z)}{z-z_0}\dif z} = \lim_{n \to \infty} \abs{\sum_{k=1}^{n} \frac{f(z_k)}{z_k-z_0}}\Delta z_k }.
\end{equation}
For this sum we will use the generalized triangle inequality, which states that
\begin{equation}
  \abs{\sum_i = a_i} \leq \sum_i \abs{a_i},
\end{equation}
which is true for an arbritary number of terms. Using this we rewrite the Riemann-sum
\begin{equation}
  \abs{\oint_{C_R} \frac{f(z)}{z-z_0}\dif z} \leq \lim_{n \to \infty} \sum_{k=1}^{n} \abs{\frac{f(z_k)}{z_k-z_0}}}\Delta z_k = \lim_{n \to \infty} \sum_{k=1}^{n} \frac{\abs{f(z_k)}}{\abs{z_k-z_0}}}\Delta z_k,
\end{equation}
where we have used that the absolute value of the fraction is just the absolute value of each factor. We have already defined $\abs{f(z)}$ in equation \eqref{M}, and we recognize the denominator as the radius $R$ from the definition of the contour \eqref{contour}. Since both of these are constants we can take them outside of the sum, where we now have a new upper bound estimate
\begin{equation}
  \abs{\oint_{C_R} \frac{f(z)}{z-z_0}\dif z} \leq \frac{M}{R} \,\,\lim_{n \to \infty} \sum_{k=1}^{n} \Delta z_k.
\end{equation}
The infinitesimal sums over the changes in $z_k$ will add up to the circomference of the curve, which is just a circle with radius $R$. Thus the upper bound can be written as
\begin{equation}
  \abs{\oint_{C_R} \frac{f(z)}{z-z_0}\dif z} \leq \frac{M}{R}\,2\pi R.
\end{equation}
We cancel the $R$'s leaving us with the simple expression for the upper bound estimate
\begin{equation}
  \abs{\oint_{C_R} \frac{f(z)}{z-z_0}\dif z} \leq 2\pi M.
\end{equation}
We insert this expression into our original one \eqref{begining}
\begin{equation}
  \abs{f(z_0)} = \frac{1}{2\pi}\,\abs{\oint_{C_R} \frac{f(z)}{z-z_0}\dif z} \leq \frac{1}{2\pi} \, 2\pi M.
\end{equation}
Canceling the factors of $2\pi$ we find the final expression for the upper bound estimate of the contour integral
\begin{equation}
  \abs{f(z_0)} \leq M.
\end{equation}

\subsubsection*{b)}
We will try to rewrite Cauchy's integral formula \eqref{CIF} for the special case of a circular contour around a point $z_0$. We do this by writing the complex number $z$ in terms of the center of the circle plus another terms looping around a cirlce with radius $R$
\begin{equation}
  z = z_0 + R e^{it} \,\,\qquad t \in [0, 2\pi].
\end{equation}
By taking the derivative of $z$ with respect to time we can solve for the infinitesimal $\dif z$
\begin{equation}
  \frac{\dif z}{\dif t} = iRe^{it} \rightarrow \dif z = iRe^{it} \dif t.
\end{equation}
We use this substitution in Cauchy's integral formula \eqref{CIF}
\begin{equation}
  f(z_0) = \frac{1}{2\pi i} \oint_{C_R} \frac{f(z)}{z-z_0}\dif z = \frac{1}{2\pi i} \int_{0}^{2\pi} \frac{f(z_0+Re^{it})}{z_0+Re^{it}-z_0}\,iRe^{it} \dif t.
\end{equation}
We see that the $i$ outside of the integral cancels with the one inside, and by subtracting away the $z_0$'s in the denominator we can cancel the factor $Re^{it}$, thus
\begin{equation}
  f(z_0) = \frac{1}{2\pi} \int_{0}^{2\pi} f(z_0+Re^{it}) \dif t. \label{answer1b}
\end{equation}
This expression will only work for circular contours around $z_0$.

\subsubsection*{c)}
We introduce the function $u(x, y)$, which is harmonic on and inside a circle of radius $R$ centerd at $z_0=x_0+iy_0$. We can evaluate the value of $u$ at the center of the circle using Cauchy's integral formula \eqref{CIF}
\begin{equation}
  f(z_0) = \frac{1}{2\pi i} \oint_{C_R} \frac{f(z)}{z-z_0}\dif z.
\end{equation}
In the previous task we showed that such a integral, for a circular contour with radius $R$ around a point $z_0$ can be rewritten to a integral over a real scalar $t$ from $0$ to $2\pi$ \eqref{answer1b}. We use this result, but now for a variable $\theta$ over the same interval, to evaluate $z_0$ at the center of the circel
\begin{equation}
  u(x_0, y_0) = \frac{1}{2\pi} \int_{0}^{2\pi} f(z_0+Re^{i\theta}) \dif \theta.
\end{equation}
This result tells us that to evalue an analytic, and in this case harmonic, function at the center of a circle we can only use values along a circle around the point. Due to the factor $1/2\pi$ outside the integral the functionvalue at the center of the circle is equal to the average value of the function along the contour. It is quite incredible that we can do this for an arbritary radius $R$ (as long as the function is still analytic inside and on the contour), and always be able to evalue the function at a point we never looked at.

\subsubsection*{d)}
For a pair of two dimensional functions $u(x, y)$ and $v(x, y)$ which are each other's harmonic conjugates we have the following relation
\begin{equation}
  \pdv{u}{x} = \pdv{v}{y} \qquad \qquad \pdv{v}{x} = - \pdv{u}{y}, \label{harmonic_def}
\end{equation}
which is actually the definition of a harmonic conjugate in two dimensions.
We can use this to derive the orthogonality of the gradients of the functions, where the two dimensional nabla-operator is defined as
\begin{equation}
  \nabla = \left( \pdv{}{x}, \,\, \pdv{}{y} \right),
\end{equation}
meaning that the inner product between the two functions can be written as
\begin{equation}
  \left( \vec{ \nabla u} \right) \,\cdot   \left( \vec{ \nabla v} \right) = \left( \pdv{u}{x}, \,\, \pdv{u}{y} \right)\cdot \left( \pdv{v}{x}, \,\, \pdv{v}{y} \right) = \pdv{u}{x}\pdv{v}{x} + \pdv{u}{y}\pdv{v}{y}.
\end{equation}
Where we have written the left hand side in bold to emphesise that it is a vector. We then make a substitution from the definition of harmonic conjugates \eqref{harmonic_def} on $v$ so that we only have derivatives acting on $u$
\begin{equation}
  \left( \vec{ \nabla u} \right) \,\cdot   \left( \vec{ \nabla v} \right) = -\pdv{u}{y}\pdv{u}{x} + \pdv{u}{x}\pdv{u}{y} = 0.
\end{equation}
Since both terms are equal, but with oposite sign, they exactly cancel each other. Thus we have showed that the gradient of two functions which are each others harmonic conjugates have to be orthogonal. We did not specify anything about $u$ and $v$ except from them being harmonic conjugates of each other, thus this orthogonality must hold for all pairs of analytical conjugates.

\subsubsection*{e)}
We will now look at concrete example where one of the harmonic functions, $u(x, y)$, is known, and we want to find it's harmonic conjugate $v(x,y)$. The expression for the known harmonic function is
\begin{equation}
  u(x,y) = \sin{x}\cosh{y}.
\end{equation}
We begin by finding it's derivatives and second derivatives with respect to both $x$ and $y$ seperately
\begin{align}
  \begin{split}
  \pdv{u}{x} &= \cos{x}\cosh{y}\qquad \qquad\pdv[2]{u}{x} = -\sin{x}\cosh{y} \\
  \pdv{u}{y} &= \sin{x}\sinh{y}\qquad \qquad\pdv[2]{u}{y} = \sin{x}\cosh{y}. \label{derivatives}
\end{split}
\end{align}
We begin by checking that $u(x, y)$ infact is harmonic
\begin{equation}
  \nabla^2u = \pdv[2]{u}{x}+\pdv[2]{u}{y} = -\sin{x}\cosh{y} + \sin{x}\cosh{y} = 0,
\end{equation}
here we used the double derivatives calculated in \eqref{derivatives}, and find that $u$ is harmonic. We now want to find the harmonic conjugate of $u$, which we can solve for through the definition of harmonic conjugates \eqref{harmonic_def}. We begin with the first equality in \eqref{harmonic_def}, and put in the derivative of $u$ from \eqref{derivatives}
\begin{equation}
  \dv{v}{y} = \dv{u}{x} = - \sin{x}\sinh{y}.
\end{equation}
We multiply each side with $\dif y$ and take the integral on both sides
\begin{equation}
  \int \dif v =  \int \cos{x}\cosh{y} \dif y.
\end{equation}
The left hand side will just be the function $v(x, y)$, while on the right hand side we can take $\cos{x}$ outside the integral, and the integral of $\cosh{y}$ is known, leaving us with
\begin{equation}
  v(x, y) =  \cos{x}\sinh{y} + C(x). \label{first}
\end{equation}
The term $C(x)$ is the integration constant, which can depend on $x$ since we took the integral over $y$. We repeat the process for the secound equality in \eqref{harmonic_def}
\begin{equation}
  \dv{v}{x} = -\dv{u}{y} = - \sin{x}\sinh{y}.
\end{equation}
We multilpy each side with $\dif x$ and take the integral
\begin{equation}
  \int \dif v = - \int \sin{x}\sinh{y} \dif x.
\end{equation}
The left hand side will just result in the function $v(x, y)$, while the right hand side is triviall, thus
\begin{equation}
  v(x, y) = \cos{x}\sinh{y} + C(y). \label{last}
\end{equation}
Where $C(y)$ is the integration constant, which can depend on $x$ since the integral was only over $y$. The two expressions we have found (\ref{first}, \ref{last}) are consistent with one another, as they should be, and we see that the integration constant can not depend on either $x$ or $y$, and must therefore just be a constant. The final expression for $u$'s harmonic conjugate is therefore
\begin{equation}
  v(x, y) = \cos{x}\sinh{y} + C. \label{v}
\end{equation}
Having found both $u(x,y)$ and $v(x,y)$ we can find the analytic function $f=u+iv$, which has the form
\begin{equation}
  f(x, y) = u(x, y) + iv(x, y) = \sin{x}\cosh{y} + i\cos{x}\sinh{y}.
\end{equation}
In our calculations we will ignore the integration constant. We begin by writing the trigonometric and hyperbolic functions on exponential form
\begin{equation}
  f(x, y) = \frac{1}{2i}\left(e^{ix}-e^{-ix}\right)\frac{1}{2}\left(e^{y}+e^{-y}\right) + i\frac{1}{2}\left(e^{ix}+e^{-ix}\right)\frac{1}{2}\left(e^{y}-e^{-y}\right).
\end{equation}
We multiply out the parenthesis in each term, and factor out $1/4$
\begin{equation}
  f(x, y) = \frac{1}{4i}\left(e^{ix-y}+e^{ix+y}-e^{-ix-y}-e^{-ix+y}\right) + i\frac{1}{4}\left(e^{ix+y}-e^{ix-y}+e^{-ix+y}-e^{-ix-y}\right).
\end{equation}
In the secound term we multiply with $i$ in the denominator and numerator, flipping the sign and making the $i$ appear in the denominator so that we can rewrite the expression to
\begin{equation}
  f(x, y) = \frac{1}{4i}\left(e^{ix-y}+e^{ix+y}-e^{-ix-y}-e^{-ix+y} - e^{ix+y}+ e^{ix-y}-e^{-ix+y}+e^{-ix-y}\right),
\end{equation}
where we see that we have four terms canceling
\begin{equation}
  f(x, y) = \frac{1}{4i}\left(e^{ix-y}+\cancel{e^{ix+y}}-\bcancel{e^{-ix-y}}-e^{-ix+y} - \cancel{e^{ix+y}}+ e^{ix-y}-e^{-ix+y}+\bcancel{e^{-ix-y}}\right),
\end{equation}
leaving us with
\begin{equation}
  f(x, y) = \frac{1}{4i}\left(2e^{ix-y} - 2e^{-ix+y}\right).
\end{equation}
We can cancel one factor of $2$ and recognize that we can use $z=x+iy$ to rewrite the exponent as $iz=ix-y$
\begin{equation}
  f(x, y) = \frac{1}{2i}\left(e^{iz} - e^{-iz}\right).
\end{equation}
We recognize this expression as the exponential form of sinus, thus
\begin{equation}
  f(x, y) = \sin{z}.
\end{equation}

\subsubsection*{f)}
%%%%%%%%%%%%%%%%%%%%%%%%%%%% Main document
%\bibliography{citations.bib}
%\bibliographystyle{plain}
\end{document}
